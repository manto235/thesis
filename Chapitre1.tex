\chapter{Introduction}
\section{Motivation et problématique}
Notre navigation sur Internet est de plus en plus scrutée par diverses organisations, que ce soient des entreprises commerciales ou des agences gouvernementales. D'ailleurs, des révélations récentes ont permis de mettre en lumière tout un système élaboré destiné à surveiller les utilisateurs d'Internet. Cependant, ce n'est pas le sujet que je vais aborder.

Je vais plutôt me focaliser sur la surveillance opérée par les entreprises commerciales. En effet, elles essaient de nous suivre à la trace afin de déterminer nos habitudes et nos préférences. Avec l'aide de ces précieuses informations, elles peuvent alors mieux nous cibler pour proposer des services et des produits qui sont censés nous intéresser davantage. D'ailleurs, certaines informations peuvent se revendre à prix d'or auprès de compagnies d'annonces publicitaires.

Afin de préserver notre vie privée, il est nécessaire de connaître comment nous sommes identifiés. Il faut également s'assurer que les outils dont nous disposons afin de nous protéger sont réellement efficaces et adaptés à notre utilisation.

\section{Pourquoi l'anonymat sur Internet est-il important ?}

\section{Méthodologie utilisée}
Afin de répondre à cette problématique, je vais tout d'abord identifier les moyens utilisés pour nous tracer. Ensuite, je vais présenter une série d'outils censés protéger notre vie privée et évaluer leur efficacité.
