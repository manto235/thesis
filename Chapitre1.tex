\chapter{Introduction}
\section{Motivation et problématique}
Notre navigation sur Internet est de plus en plus scrutée par diverses organisations, que ce soient des entreprises commerciales ou des agences gouvernementales. D'ailleurs, des révélations récentes \cite{Wiki_RES} ont permis de mettre en lumière un système élaboré destiné à surveiller les utilisateurs d'Internet.

Ce travail se focalise et analyse la surveillance opérée par les entreprises commerciales. En effet, elles essaient de nous suivre à la trace afin de déterminer nos habitudes et nos préférences. Avec l'aide de ces précieuses informations, elles peuvent alors mieux cibler nos besoins et nous proposer des biens et des services qui sont censés nous intéresser davantage, voire anticiper nos intentions\cite{MD1}.

Cette pratique est de plus en plus répandue sur les sites de vente en ligne, grâce à l'historique de nos achats.

D'autres formes de surveillance plus vicieuses existent également. Imaginons que vous souhaitiez voyager vers un pays exotique et que vous consultez le prix des billets d'avion vers cette destination sur le site d'une compagnie aérienne. Lors d'une consultation ultérieure, le prix a augmenté et vous vous empressez alors d'acheter le billet craignant que son prix ne continue d'augmenter. Malheureusement pour vous, celui-ci a en fait été gonflé pour vous pousser à l'achat.

Les plus grosses régies publicitaires sont installées sur un grand nombre de sites et lors de chaque visite, elles enregistrent les traces que vous laissez. En regroupant toutes les informations disséminées sur ces différents sites, elles sont alors en mesure d'établir votre profil. Les détails de celui-ci peuvent se revendre à prix d'or auprès d'autres compagnies. Il est évidemment plus facile de cibler un segment consitué de personnes dont on connaît le profil socio-démographique, le sexe, l'âge, les intérêts,...

Avez-vous déjà remarqué que le contenu des publicités s'adaptait en fonction de vos requêtes effectuées sur les sites de recherche ou des sites que vous visitez ?

Comme vous pouvez le remarquer avec ces différents exemples de la vie quotidienne, la surveillance des utilisateurs d'Internet peut amener à plusieurs inconvénients qui nous touchent directement. Voici donc pourquoi il est intéressant de s'intéresser à cette surveillance et d'essayer de déterminer via quels moyens celle-ci est opérée. Afin de préserver notre vie privée sur Internet, il est nécessaire de connaître comment nous sommes identifiés. Il faut également s'assurer que les outils dont nous disposons afin de nous protéger soient réellement efficaces et adaptés à nos besoins.

\section{Pourquoi vouloir l'anonymat sur Internet ?}

\section{Méthodologie utilisée}
Afin de répondre à cette problématique, il faut tout d'abord comprendre Internet fonctionne et en particulier, certains de ses mécanismes tels que les cookies.

Ensuite, il faut identifier la manière dont ces mécanismes peuvent être utilisés ou détournés afin de permettre le traçage des utilisateurs.

Grâce à ces connaissances, nous serons en mesure de déterminer quels sont les moyens effectivement utilisés par les sites qui pratiquent cette surveillance et d'estimer son étendue sur un panel consititué des principaux sites web (classement Alexa\cite{AlexaTop}).

Pour terminer, nous pourrons déterminer si les outils censés protéger notre vie privée sont réellement efficaces. % et par quels moyens nous pourrions les améliorer.
