\chapter{Introduction}
\section{Motivation et problématique}
Notre navigation sur Internet est de plus en plus scrutée par diverses organisations, que ce soient des entreprises commerciales ou des agences gouvernementales. D'ailleurs, des révélations récentes ont permis de mettre en lumière tout un système élaboré destiné à surveiller les utilisateurs d'Internet. Cependant, ce n'est pas le sujet que je vais aborder.

Je vais plutôt me focaliser sur la surveillance opérée par les entreprises commerciales. En effet, elles essaient de nous suivre à la trace afin de déterminer nos habitudes et nos préférences. Avec l'aide de ces précieuses informations, elles peuvent alors mieux nous cibler pour nous proposer des biens et des services qui sont censés nous intéresser davantage ou deviner nos intentions.

C'est par exemple ce que pratiquent plusieurs sites de ventes en ligne\cite{MD1}. Avec l'historique de vos achats, ces entreprises sont capables de déterminer les produits qui pourraient susciter votre intérêt. D'autres formes de surveillance plus vicieuses existent également. Imaginons que vous souhaitiez voyager vers un pays exotique et que vous regardez le prix des billets d'avion vers cette destination sur le site d'une compagnie aérienne. Vous y retournez plus tard et à votre désarroi, vous constatez que le prix a augmenté. Vous vous empressez alors d'acheter le billet craignant que son prix ne continue d'augmenter. Malheureusement pour vous, celui-ci a en fait été gonflé pour vous pousser à l'achat. En effet, le site vous a reconnu et afin de convaincre à acheter le billet, il a augmenté le prix par rapport à votre précédente visite. Je vais clôturer cette petite mise en situation avec une simple question: avez-vous déjà remarqué que le contenu des publicités s'adaptait en fonction de vos requêtes effectuées sur les sites de recherche et de vos sites visités ?

Comme vous pouvez le remarquer avec ces différents exemples de la vie quotidienne, la surveillance des utilisateurs d'Internet peut amener à plusieurs inconvénients qui nous touchent directement. Voici donc pourquoi il est intéressant de s'intéresser à cette surveillance et d'essayer de déterminer via quels moyens celle-ci est opérée.
%D'ailleurs, certaines informations peuvent se revendre à prix d'or auprès de compagnies d'annonces publicitaires.

%Afin de préserver notre vie privée sur Internet, il est nécessaire de connaître comment nous sommes identifiés. Il faut également s'assurer que les outils dont nous disposons afin de nous protéger soient réellement efficaces et adaptés à notre utilisation.

\section{Pourquoi vouloir l'anonymat sur Internet ?}

\section{Méthodologie utilisée}
Afin de répondre à cette problématique, je vais tout d'abord identifier les moyens utilisés pour nous tracer. Ensuite, je vais présenter une série d'outils censés protéger notre vie privée et évaluer leur efficacité.
