\chapter{Moyens d'identification généralement utilisés}
\section{Cookies}
En ce qui concerne les cookies, il est assez simple de tracer la navigation d'un utilisateur. En effet, comme expliqué auparavant, le navigateur envoie dans chaque requête un en-tête \textit{Cookie} et il est donc aisé pour le site web de suivre le parcours du visiteur simplement en analysant cet en-tête. Le fait de passer par un proxy ne change rien car le Cookie reste indépendant du moyen d'accès au site web.
\section{Pixels espions}
\section{JavaScript}
	\subsection{Régies publicitaires sur Internet}
		
	\subsection{Modules des réseaux sociaux}
		
		\subsubsection{Facebook et son bouton « Like »}
			
		\subsubsection{Google et son bouton « +1 »}
			
	\subsection{Outils destinés aux webmasters}
		
		\subsubsection{Google Analytics}

\section{Autres moyens de nous tracer}
	\subsection{Flash}
	
	\subsection{Java}
	
	\subsection{Omniprésence des navigateurs dans les applications}
		
	\subsection{Persistance des données utilisateur dans Microsoft Internet Explorer}
		
	%\subsection{Identifiant unique que Google envisage d'implémenter}
	