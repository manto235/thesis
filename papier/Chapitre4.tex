\chapter{Analyse des principaux sites web}
\section{Objectif de l'analyse}
Le but de l'analyse est de parcourir les sites les plus visités au monde d'après le classement Alexa \cite{AlexaTop}. Ensuite, déterminer s'ils contiennent des trackers et, le cas échéant, leur nature. Deux types d'expériences peuvent être menés :
\begin{enumerate}
	\item Le premier consiste en une analyse à long terme.
	\item Le second consiste en une analyse ponctuelle.
\end{enumerate}

Le premier type d'expérience a pour but de déterminer si les sites modifient leurs trackers au fil du temps.
Le second type permet d'avoir une image globale du niveau de traçage opéré par les principaux sites à un instant donné. Il donne également la possibilité de tester l'efficacité des extensions de navigateurs qui affirment protéger la vie privée (voir \autoref{results_plugins}).

\section{Description de l'outil implémenté}
L'idée initiale était d'utiliser le framework \textit{fpdetective} \cite{Acar:2013:FDW:2508859.2516674}. C'est un framework conçu pour la détection et l'analyse d'outils (\textit{fingerprinters}, voir \autoref{fingerprinters}) qui créent des empreintes de navigateurs. Ce framework semblait prometteur mais le problème est qu'il ne fonctionnait pas correctement. En effet, aucun \textit{fingerprinter} n'était détecté alors qu'ils étaient manifestement présents sur plusieurs sites web visités.

Etant donné que \textit{fpdetective} ne pouvait fournir les résultats demandés, la création d'un outil dédié à cette tâche était nécessaire. De plus, cela permettait plus de souplesse au niveau des décisions d'implémentation et du paramétrage de la recherche de trackers.
\newline

L'outil a été implémenté en Java. Il contient deux éléments distincts : le \textit{crawler} qui visite les sites web et exporte leur contenu au format HTTP Archive (HAR) et le \textit{parser} qui traite les fichiers HAR en déterminant si les sites correspondants renferment des trackers.
L'outil a été développé sous Eclipse et est destiné à être utilisé sur Linux. Un paquage Jar est également disponible.
\newline

Lors du lancement du \textit{crawler}, l'utilisateur peut spécifier plusieurs arguments :
\begin{itemize}
	\item Requis : le mode (\textit{crawler} ou \textit{parser}).
	\item Requis : le répertoire des fichiers.\\
		Pour le \textit{crawler}, il s'agit du répertoire où les fichiers seront enregistrés.\\
		Pour le \textit{parser}, il s'agit du répertoire contenant les fichiers à analyser.
	\item Optionnel : l'activation du mode \textit{debug} qui va donner davantage de détails en cas de problème.
	\item Optionnel : l'affichage de l'aide.
	\newline
	\item Requis pour le \textit{crawler} : le profil de Firefox à utiliser.
	\item Requis pour le \textit{crawler} : le fichier contenant la liste des sites web à visiter.
	\item Requis pour le \textit{crawler} : le début de l'intervalle des sites à visiter.
	\item Requis pour le \textit{crawler} : la fin de l'intervalle des sites à visiter.
	\item Requis pour le \textit{crawler} : le nombre maximal de tentatives par site web (\textit{timeout}).
	\item Requis pour le \textit{crawler} : le nombre de sites à visiter avant le redémarrage de Firefox.
	\newline
	\item Requis pour le \textit{parser} : le fichier contenant la liste des trackers de Ghostery.
	\item Optionnel pour le \textit{parser} : l'activation de l'impression de tous les trackers identifiés.
\end{itemize}

\subsection{Crawler}
\label{crawler}
\subsubsection{Implémentation}
Afin d'automatiser le navigateur et de parcourir les sites sans intervention humaine, \textit{Selenium} \cite{selenium_homepage} est utilisé. Plus précisément, c'est le pilote \textit{FirefoxDriver} de \textit{Selenium} qui est utilisé au sein de l'outil.

La première implémentation traitait directement le code source de la page. Il était possible d'aller sur le moteur de recherche Google, de sélectionner le champ de recherche, de taper une requête et de récupérer le résultat. Cependant, le processus était assez lent et incomplet car on ne pouvait pas récupérer les requêtes et les réponses HTTP. On pouvait juste récupérer les éléments associés à des balises (par exemple, récupérer toutes les images via les balises de type \textit{<img>}, les scripts avec les balises \textit{<script>},...) mais tout n'était pas récupérable car certains éléments étaient chargés sans être identifiés.

Afin de palier ce problème, la deuxième implémentation utilisait un proxy qui exportait le contenu du site au format HTTP Archive. L'ensemble des requêtes et réponses HTTP était récupéré et enregistré dans le fichier. Le parcours du code source (qui donnait des résultats finalement incomplets) a été supprimé afin de rendre le programme plus rapide. Cette fois, le problème était que le proxy n'exécutait pas le JavaScript. Il en résultait que de nombreux sites (cela pouvait varier de 15 à 30\% d'échecs) ne pouvait être chargés correctement et il était alors impossible de les traiter.

La troisième solution, qui est la dernière implémentée, consiste à utiliser le pilote pour charger les sites automatiquement, sans parcours du code source ni utilisation de proxy. Le contenu des sites est exporté au format HTTP Archive à l'aide de deux extensions qui ont été installées dans Firefox.
\begin{itemize}
	\item La première, Firebug \cite{firebug_homepage}, est une extension utilisée principalement par les développeurs de sites web afin de les débugger. Elle permet de monitorer les éléments d'une page (CSS, HTML, JavaScript) en direct.
	\item La seconde, NetExport \cite{netexport_homepage}, est une extension de Firebug qui permet d'exporter toutes les données d'une page au format HTTP Archive de façon automatique (il est possible de paramétrer cette extension pour le faire).
\end{itemize}

Le taux de réussite de l'export des données des sites s'est grandement amélioré. Néanmoins, certains sites ne sont pas chargés correctement.
Une hypothèse est que certains sites sont hébergés sur des serveurs lointains et les éléments prennent plus de temps à charger. Ainsi, lorsque ces sites contiennent une grande quantité d'éléments, le pilote renvoie une exception de type \textit{timeout} et le chargement du site est finalement en échec. On remarque généralement ceci sur les sites asiatiques : certains portails regorgent d'énormément de contenus qui prennent du temps à charger.
Une hypothèse supplémentaire est que certains sites demandent une action de l'utilisateur (faire défiler la page par exemple) afin de charger les éléments du site (typiquement, les images).

\subsubsection{Fonctionnement}
Le \textit{crawler} fonctionne de la manière suivante : tout d'abord, il vérifie que le répertoire donné en argument est accessible en écriture et il crée le fichier de log. Ensuite, il charge la liste des sites à visiter depuis le fichier spécifié en fonction de l'intervale donné lors du lancement du programme. Puis, le profil Firefox choisi est chargé et différents paramètres de configuration sont également réglés dans les deux extensions avant que le pilote soit initialisé et qu'il lance Firefox. Après le lancement du pilote, un délai de 5 secondes laisse le temps à Firefox de se lancer et de charger ses extensions.
\newline

Ensuite, le parcours des sites commence. Le pilote dirige Firefox sur l'URL du premier site de la liste. Lorsque le chargement de la page est terminé, un délai de 8 secondes permet aux extensions d'enregistrer le fichier HTTP Archive.

Si la page s'est chargée avant la limite de 30 secondes, l'enregistrement est considéré comme valide et le pilote charge le site suivant dans la liste.

Si un site fait un timeout et que l'utilisateur a précisé un nombre de tentatives supérieur à 1, le pilote dirige Firefox sur la page "about:blank" (cela permet d'éviter des problèmes d'écriture de fichiers avec les extensions) puis le redirige sur le site pour un nouvel essai. Ce processus est répété à chaque timeout jusqu'à atteindre le nombre maximal de tentatives. Si le site fait encore un timeout lors de la dernière tentative, il est alors considéré en échec et son URL est inscrite dans une liste reprenant tous les sites en échec. A la deuxième tentative, l'URL du site est également enregistrée dans une autre liste reprenant tous les sites ayant fait un timeout, cela permet de garder une trace des sites dont la visite a rencontré un problème mais qui ont finalement pu être chargés. 

Il arrive également qu'un site ne puisse tout simplement pas être chargé (à cause d'une erreur dans son code source par exemple). Son URL est alors directement enregistrée dans la liste reprenant les sites en échec.
\newline

Lorsque le parcours de l'ensemble des sites est terminé, le programme efface les fichiers inutiles générés lors de la visite des pages "about:blank" et détaille l'ensemble des sites ayant subi au moins un timeout et les sites en échec. Pour terminer, le pilote est arrêté (ce qui entraîne la fermeture de Firefox) et le fichier de log est fermé.
\newline

Il a été nécessaire d'implémenter une fonctionnalité qui redémarre Firefox à intervales réguliers. En effet, certains sites ouvrent des pop-ups et si le navigateur n'est pas fermé, ces derniers s'accumulent et occupent une part plus importante de ressources. Le redémarrage du navigateur permet de fermer toutes ces fenêtres ouvertes de façon intempestive afin de garantir une meilleure stabilité du \textit{crawler}.

\subsubsection{Changements opérés sur les extensions}
Deux modifications ont été nécessaires sur l'extension \textit{NetExport} de \textit{Firebug}.

La première concerne la génération des fichiers HTTP Archive. Par défaut, l'extension ajoute un champ qui n'est pas standard par rapport à la spécification du format HAR. Il en résulte que les fichiers peuvent alors être considérés comme corrompus alors que la présence ou non de ce champ n'influence en rien les résultats qui nous concernent. La génération de ce champ a donc été supprimée dans le fichier \textit{/chrome/content/netexport/harBuilder.js} de \textit{NetExport}.

La seconde modification concerne le nom des fichiers générés. Par défaut, l'extension ajoute la date à l'URL du site dans les noms de fichier. Cela pose des difficultés supplémentaires au \textit{parser} lors de la récupération et du traitement de ces fichiers (voir \autoref{parser}). L'extension a donc été modifiée afin de n'utiliser que l'URL du site dans les noms du fichier. Cette modification a été effectuée dans le fichier \textit{/chrome/content/netexport/automation.js}. Sur Linux, lorsqu'un fichier de même nom existe déjà, un tiret suivi du numéro de copie est ajouté au nom de fichier avant son extension.

\subsection{Parser}
\label{parser}

\section{Résultats}

\subsection{Expérience 1 : étude à long terme}
Afin de réaliser cette étude, le \textit{crawler} a été lancé chaque jour du 28 mars au 7 mai 2014 (à l'exception du 8 avril) sur le TOP 1000 du classement Alexa \cite{AlexaTop}, ce qui représente 40 jours de mesures.

\subsection{Expérience 2 : étude ponctuelle}

\subsubsection{Sites renfermant le plus de trackers détectés}

\subsubsection{Organisations déployant le plus de trackers}
