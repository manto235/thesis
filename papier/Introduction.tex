\chapter{Introduction}
\section{Motivation et problématique}
Notre navigation sur Internet est de plus en plus scrutée par diverses organisations, que ce soient des entreprises commerciales ou des agences gouvernementales. D'ailleurs, des révélations récentes d'Edward Snowden \cite{WikipediaFR_RES}, un informaticien américain et ancien employé de la CIA et de la NSA, ont permis de mettre en lumière un système élaboré destiné à surveiller les utilisateurs d'Internet.

Ce travail se focalise sur la surveillance opérée par les entreprises commerciales. En effet, elles essaient de nous suivre à la trace afin de déterminer nos habitudes et nos préférences. Avec l'aide de ces précieuses informations, elles peuvent alors mieux cibler nos besoins et nous proposer des biens et des services qui sont censés nous intéresser davantage, voire anticiper nos intentions\cite{MD1}.

Cette pratique est de plus en plus répandue sur les sites de vente en ligne, grâce à l'historique de nos consultations et achats.

D'autres formes de surveillance plus vicieuses existent également. Imaginons que vous souhaitiez voyager vers un pays exotique et que vous consultez le prix des billets d'avion vers cette destination sur le site d'une compagnie aérienne. Lors d'une consultation ultérieure, le prix a augmenté et vous vous empressez alors d'acheter le billet craignant que son prix ne continue d'augmenter. Malheureusement pour vous, celui-ci a en fait été gonflé pour vous pousser à l'achat.

Les plus grosses régies publicitaires sont installées sur un grand nombre de sites et lors de chaque visite, elles enregistrent les traces que vous laissez. En regroupant toutes les informations disséminées sur ces différents sites, elles sont alors en mesure d'établir votre profil. Les détails de celui-ci peuvent se revendre à prix d'or auprès d'autres compagnies. Il est évidemment plus facile de cibler un segment constitué de personnes dont on connaît le profil sociodémographique, le sexe, l'âge, les intérêts,...

Avez-vous déjà remarqué que le contenu des publicités s'adaptait en fonction de vos requêtes effectuées sur les moteurs de recherche ou des sites que vous visitez ?

Comme vous pouvez le remarquer avec ces différents exemples de la vie quotidienne, la surveillance des utilisateurs d'Internet peut générer plusieurs inconvénients qui nous touchent directement. Il est donc intéressant d'analyser cette surveillance et déterminer par quels moyens elle s'opère. Afin de préserver notre vie privée sur Internet, il est nécessaire de connaître comment nous sommes identifiés. Il faut également s'assurer que les outils dont nous disposons afin de nous protéger soient réellement efficaces et adaptés à nos besoins.


%%%%%%%%%%%%%%%%%%%%%%%%%%%%%%
\section{La vie privée}
Le respect de la vie privée est intimement lié à l'évolution de la surveillance.
Il y a toujours eu des conflits entre les personnes qui estiment que tout le monde doit pouvoir utiliser et partager une technologie permettant de sécuriser ses communications et les gouvernements qui souhaiteraient garder la possibilité d'écouter ces communications. Prenons l'exemple du gouvernement américain qui avait peur de passer d'un monde où il avait un contrôle total à un monde où il n'en avait plus. En conséquence, il a considéré l'export de la cryptographie hors des USA équivalent à l'export de munitions. Ainsi, afin de dévoiler les sources de PGP (Pretty Good Privacy, un logiciel de chiffrement et de déchiffrement cryptographique) au monde entier, son auteur a usé d'une astuce en publiant un livre qui en contient le code source complet (l'export de livres est protégé par le Premier Amendement) \cite{youtube_moxie_marlinspike}.
\newline

Les programmes de surveillance du passé peuvent être considérés comme de la "surveillance de proximité", où un gouvernement tentait d'utiliser la technologie afin de surveiller les communications lui-même. Les programmes actuels marquent une transition vers la "surveillance oblique" dans laquelle un gouvernement va plus souvent se rendre aux endroits où les informations sont accumulées (fournisseurs d'adresses e-mail, moteurs de recherche, réseaux sociaux et télécoms) \cite{wired_nothing_to_hide}.
\newline

En 2001, John Poindexter, alors qu'il occupait un poste officiel au sein du Département de la Défense des États-Unis, a voulu mettre en place le programme \textit{Total Information Awareness} qui consistait en un système de data mining. Celui-ci devait enregistrer tous les e-mails, tout le trafic web, tout l'historique des cartes de crédit et tous les dossiers médicaux. Le but était ensuite de développer une technologie afin d'extraire les données dont on avait besoin.

Lors de sa conférence à la DEF CON 18, Moxie Marlinspike \footnote{Moxie Marlinspike est le pseudonyme d'un chercheur en sécurité informatique qui est engagé dans le respect de la vie privée \cite{site_perso_moxie_marlinspike}.} a tourné en ridicule le logo du programme (\autoref{IAO_logo}) en expliquant qu'il était déconseillé d'utiliser un logo qui inspirait la peur. D'ailleurs, le projet a été annulé car il a reçu beaucoup de contestations. Cependant, lorsque l'on regarde ce que Google fait aujourd'hui, c'est exactement la même chose, voire pire. Or, personne ne proteste et de surcroît, tout le monde utilise ses services !
\begin{itemize}
  \item Les emails sont enregistrés par Gmail.
  \item Le trafic web est enregistré par Google Analytics.
  \item L'historique des cartes de crédit est enregistré par Google Checkout.
  \item Les dossiers médicaux sont enregistrés par Google Health.
  \item L'historique des positions GPS est enregistré par Android.
  \item ...
  %\newline
\end{itemize}
De plus, on sait que Google a la capacité d'extraire de manière efficace les informations qu'il désire afin de les monétiser avec des publicités. Lorsque le CEO de Google, Eric Schmidt, déclare : "If you have something that you don't want anyone to know, maybe you shouldn't be doing it in the first place" \cite{privacy_eric_schmidt}, cela fait froid dans le dos.

\begin{figure}[h]
	\centering
	\includegraphics[scale=0.2]{figures/IAO-logo.png}
	\caption{\label{IAO_logo}Le logo officiel du programme \textit{Total Information Awareness}.}
\end{figure}

Dans sa conférence, Moxie Marlinspike utilise un autre exemple : personne n'accepterait de porter un tracker monitoré par l'État mais tout le monde possède et porte un GSM en permanence. Or, celui-ci envoie en temps réel sa position à l'opérateur qui est obligé de délivrer cette information si on la lui demande. Pour lui, il y a cependant une différence essentielle : le choix. Nous choisissons de posséder un GSM afin de communiquer mais nous refuserions un dispositif de traçage s'il nous était imposé.
\newline

%%%%%
\newpage
Maintenant, passons au sujet de la vie privée.\\
D'après Moglen \footnote{Eben Moglen est professeur de droit et d'histoire du droit à l'Université Columbia. Il est le président du Software Freedom Law Center, une organisation qui procure assistance et défense juridique aux développeurs de logiciels libres et open-source.} \cite{Moglen_part2}, le concept de vie privée englobe 3 éléments.
\begin{itemize}
  \item La confidentialité : la capacité de garder des messages privés c'est-à-dire que leur contenu n'est connu que de ceux à qui ils sont destinés.
  \item L'anonymat : la capacité d'envoyer des messages sans savoir de qui ils proviennent ni à qui ils sont destinés, même si leur contenu est ouvert.
  \item L'indépendance : la capacité de prendre des décisions librement, sans que la confidentialité ou l'anonymat ne soient violés.
\end{itemize}

Pour lui, si l'un de ces éléments n'est pas respecté, on ne peut pas avoir de gouvernement démocratique car la vie privée en est une condition nécessaire.
\newline

Il explique également qu'on ne devrait pas voir la vie privée comme une transaction. En effet, ceux qui veulent profiter de nos informations souhaitent définir la vie privée comme un concept que l'on peut négocier. Ainsi, ils nous offrent un service gratuit comme l'accès à une messagerie mais en échange, ils lisent nos mails. Pour eux, il s'agirait seulement d'une transaction entre deux parties. Cependant, si on y réfléchit, ce n'est pas une transaction quelconque car tous ceux qui nous écrivent sont aussi impliqués dans cet accord, qui était supposé être bilatéral.

Rien n'est gratuit dans la vie et cela l'est également sur Internet. En réalité, le prix à payer est de dévoiler des informations personnelles qui peuvent potentiellement être exploitées afin de rapporter de l'argent.
\newline

Les exemples de l'introduction montrent déjà clairement que la surveillance des sites web peut nous coûter de l'argent en nous incitant à consommer plus. Mais les risques de cette surveillance peuvent aller bien au-delà. En effet, la liste des sites web visités par une personne peut en révéler beaucoup sur sa situation familiale, financière ou médicale.
\newline

Imaginons une situation où le directeur des ressources humaines d'une entreprise reçoit des CV. Il décide de trouver davantage d'informations au sujet des candidats et pour cela, tape leur nom dans un moteur de recherche. Il pourrait tomber sur la photo d'une jeune femme portant un bébé dans ses bras ou sur la photo d'une jeune femme seule. Pensez-vous qu'elles aient la même chance d'être rappelées pour un entretien ?
\newline

Lors d'une expérience menée avec son équipe \cite{TED_Alessandro_Acquisti}, Alessandro Acquisti \footnote{Alessandro Acquisti est un économiste comportemental à l'Université Carnegie Mellon et il étudie l'économie comportementale de la vie privée.} a créé des profils Facebook en modifiant certaines de leurs caractéristiques. Il a ensuite envoyé des candidatures à différentes entreprises américaines et a analysé leur comportement face aux profils associés aux candidatures. Les résultats de l'expérience ont montré que les entreprises agissaient en fonction des informations qu'elles avaient trouvées sur les réseaux sociaux.
\newline

Face à la montée de la surveillance, certaines personnes émettent le désir de préserver leur vie privée. On leur rétorque souvent "Si vous ne faites rien de mal, qu'avez-vous à cacher ?".
D'après Bruce Schneier \footnote{Bruce Schneier est un cryptologue, spécialiste en sécurité informatique et écrivain américain.} \cite{the_eternal_value_of_privacy}, la vie privée n'a rien à voir avec le fait d'avoir quelque chose de négatif à cacher, c'est un besoin humain essentiel.
Nous faisons énormément de choses en privé qui ne sont pas considérées comme illégales mais que nous préférons garder pour nous. Mettre des rideaux à ses fenêtres n'est pas criminel, cela traduit juste notre envie de garder notre intimité, vie privée.
Ce qui l'ennuie c'est qu'on présuppose que la vie privée sert à dissimuler des mauvaises actions alors que ce n'est pas le cas. Il affirme même que le droit à une vie privée est un droit de l'être humain. D'ailleurs, lorsque la Constitution américaine fut rédigée, le principe de vie privée n'y a pas été inscrit car il était considéré comme inhérent aux êtres. A l'époque, on ne surveillait que les criminels, pas les honnêtes citoyens.

Lorsqu'une personne se sent surveillée, elle n'agit plus de la même façon. Cela vous est sûrement déjà arrivé de cesser votre conversation si vous vous sentez observés, par peur que vos mots ne soient sortis de leur contexte. Une surveillance généralisée a le même effet mais sur un ensemble de personnes beaucoup plus étendu.
\newline

Est-ce que les personnes déclarant ne rien avoir à cacher resteraient du même avis si leur gouvernement, avec l'aide de toutes les informations qu'il possède grâce à la surveillance, décidait que leurs actions étaient suspectes et que des sanctions étaient appliquées arbitrairement ?\\
Une simple question pourrait également les raisonner : donneriez-vous votre adresse, numéro de téléphone, numéro de carte de crédit et d'autres données personnelles à un inconnu dans la rue ?

%%%%%%%%%%%%%%%%%%%%%%%%%%%%%%
\section{Méthodologie utilisée}
Afin de répondre à la problématique de la surveillance sur Internet, nous voulons tout d'abord comprendre comment Internet fonctionne et en particulier, certains de ses mécanismes tels que les cookies.

Ensuite, nous voulons identifier la manière dont ces mécanismes peuvent être utilisés ou détournés afin de permettre le traçage des utilisateurs.

Grâce à ces connaissances, nous serons en mesure de détecter quels sont les moyens effectivement utilisés par les sites qui pratiquent cette surveillance et d'estimer son étendue sur un panel constitué des principaux sites web visités au monde.

Pour terminer, nous pourrons déterminer si les outils censés protéger notre vie privée sont réellement efficaces. % et par quels moyens nous pourrions les améliorer.
\newline

Les sources ainsi que l'exécutable de l'outil développé sont disponibles publiquement sur \url{https://github.com/manto235/thesis_final}. Une version électronique du mémoire y est également présente.
