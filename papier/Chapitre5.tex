\chapter{Moyens de défense}
\section{Extensions des navigateurs}
Certaines extensions de navigateurs ont été developpées afin de préserver la vie privée des utilisateurs.
Afin de tester les différentes extensions de ce type disponibles pour Firefox, chacune d'elles a été installée dans un profil Firefox spécifique et l'outil a été lancé afin de réaliser une expérience ponctuelle.

\subsection{Ghostery}
\subsubsection{Présentation}
Ghostery \cite{ghostery_homepage} est une extension disponible pour Firefox, Google Chrome, Opera et Safari. Son fonctionnement repose sur une base de données de trackers alimentée par les retours des utilisateurs de l'extension. Evidon, la société qui possède Ghostery, a fait momentanément parler d'elle car elle a des contrats avec certaines entreprises. Certains lui reprochaient alors de vendre les informations reçues des utilisateurs via son programme Ghostrank. Sur leur site, les développeurs de Ghostery assurent ne pas vendre d'informations personnelles.

\subsubsection{Résultats}
L'extension a été activée avec tous les trackers et tous les cookies sélectionnés. Le \textit{crawler} a ensuite été lancé sur le TOP 1000 du classement Alexa \cite{AlexaTop} du 16 mai 2014.

\subsection{Adblock}
\subsubsection{Présentation}
\subsubsection{Résultats}

\subsection{Priv3}
\subsubsection{Présentation}
\subsubsection{Résultats}

\subsection{Privacy Badger}
\subsubsection{Présentation}
\subsubsection{Résultats}

\subsection{DoNotTrackMe}
\subsubsection{Présentation}
\subsubsection{Résultats}

\subsection{BetterPrivacy}
\subsubsection{Présentation}
\subsubsection{Résultats}

%\section{Modes de navigation privée}
\section{Do Not Track}
\subsection{Présentation}
\subsection{Résultats}

%\section{Utilisation d'un proxy / VPN}

\section{Résultats}
\label{results_plugins}

%\section{Les réseaux de type TOR}
%(Parler de la récente affaire du FBI qui a cracké TOR)
