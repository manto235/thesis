\chapter{Protocoles et mécanismes d'Internet}
\section{Présentation du protocole HTTP}
Le protocole HTTP fournit les fondations du World Wide Web. Lorsqu'un utilisateur clique sur un lien hypertexte via son navigateur, celui-ci se connecte à un serveur identifié par l'URL « Uniform Resource Locator » présente dans ce lien. Le navigateur envoie alors une requête à laquelle le serveur répond puis le navigateur se déconnecte du serveur.

L'on considère que la requête est sans état car à chaque fois que le navigateur crée une connexion pour une requête, le serveur la traite comme si c'était la première ; les requêtes sont donc indépendantes.

Afin d'avoir la possibilité de garder une certaine quantité d'informations entre les requêtes successives, une solution a été trouvée : il s'agit du cookie.

Les requêtes HTTP se composent de trois parties :
\begin{enumerate}
	\item Une ligne de requête
	\item Les en-têtes de requête (qui fournissent des méta-informations)
	\item L'entité de requête elle-même
\end{enumerate}

Ce sont les méta-informations de l'en-tête qui fournissent à la fois les informations de contrôle pour HTTP et les informations à propos de l'entité qui est transférée. Les informations sur les cookies sont transmises dans ces en-têtes.
	
\section{Cookies}
Dans sa réponse à une requête, un serveur peut envoyer des informations arbitraires (le cookie) dans un en-tête de réponse \textit{Set-Cookie}. Cette information peut contenir n'importe quoi. C'est elle qui permet au serveur de continuer là où il en était. Il peut s'agir d'un identifiant relatif à l'utilisateur, une clé dans une base de données,...

Habituellement, le client est coopératif et renvoie l'information du cookie qui est stocké sur le disque dur de l'ordinateur. A chaque requête qu'il fait vers le même serveur, le client indique cette information dans un en-tête \textit{Cookie}. Le serveur peut choisir de renvoyer un nouveau cookie dans ses réponses, ce qui remplacera automatiquement l'ancien cookie.

Il y a donc un contrat implicite entre le client et le serveur : ce dernier repose sur le client afin de sauvegarder son état et il espère qu'il lui sera retourné à la prochaine visite.

Un cookie est donc une donnée que le serveur et le client se renvoient l'un et l'autre. La quantité d'informations est généralement petite et son contenu est à la discrétion du serveur. En effet, la plupart du temps, analyser le contenu du cookie ne révèle ni à quoi il est destiné, ni la valeur qu'il représente.

\section{Cache}

\section{Comment le Web est-il censé fonctionner ?}

\section{Comment ces mécanismes permettent-ils de nous identifier ?}