\appendix
\chapter{Options de l'outil implémenté}
\label{options_outil_implémenté}
Lors du lancement du \textit{crawler}, l'utilisateur peut spécifier plusieurs arguments :
\begin{itemize}
	\item Requis : le mode (\textit{crawler} ou \textit{parser}).
	\item Requis : le répertoire des fichiers.\\
		Pour le \textit{crawler}, il s'agit du répertoire où les fichiers seront enregistrés.\\
		Pour le \textit{parser}, il s'agit du répertoire contenant les fichiers à analyser.
	\item Optionnel : l'activation du mode \textit{debug} qui va donner davantage de détails en cas de problème.
	\item Optionnel : l'affichage de l'aide.
	\newline
	\item Requis pour le \textit{crawler} : le profil de Firefox à utiliser.
	\item Requis pour le \textit{crawler} : le fichier contenant la liste des sites web à visiter.
	\item Requis pour le \textit{crawler} : le début de l'intervalle des sites à visiter.
	\item Requis pour le \textit{crawler} : la fin de l'intervalle des sites à visiter.
	\item Requis pour le \textit{crawler} : le nombre de sites à visiter entre chaque redémarrage de Firefox.
	\item Optionnel pour le \textit{crawler} : le nombre maximal de tentatives par site web (lors d'un échec dû à un \textit{timeout}). Par défaut : 1 tentative.
	\item Optionnel pour le \textit{crawler} : la durée du timeout lors du chargement d'un site. Par défaut : 30 secondes.
	\newline
	\item Optionnel pour le \textit{parser} : le fichier contenant la liste des trackers de Ghostery.
	\item Optionnel pour le \textit{parser} : l'affichage dans le terminal de tous les trackers identifiés pour chaque fichier analysé.
\end{itemize}

\chapter{Modifications dans l'extension NetExport}
\label{modifications_extensions}
\section*{harBuilder.js}
\label{harBuilder}
Lignes 200 à 211 commentées :
\begin{lstlisting}[frame=single]
/*if (!timings._timeStamps)
{
  timings.comment = "_timeStamps field contains timing data generated using " + "console.timeStamp() method. See Firebug documentation: " + "http://getfirebug.com/wiki/index.php/Console_API";
  timings._timeStamps = [];
}

timings._timeStamps.push({
  time: stamp.time - this.startedDateTime,
  label: label
});*/
\end{lstlisting}

\section*{automation.js}
\label{automation}
Ligne 171 modifiée :
\begin{lstlisting}[frame=single]
var fileName = name; /*+ "+" + now.getFullYear() + "-" + f(now.getMonth()+1) + "-" + f(now.getDate()) + "+" + f(now.getHours()) + "-" + f(now.getMinutes()) + "-" + f(now.getSeconds());*/
\end{lstlisting}


\chapter{Format des résultats}
\label{format_resultats}
Il existe deux types de résultats : les résultats détaillés pour chaque fichier HTTP Archive analysé et les résultats globaux de l'analyse.

Les résultats globaux sont enregistrés dans le dossier "logs" alors que les résultats détaillés sont enregistrés dans le dossier "results".
Ceci a été décidé par souci de simplicité car les résultats globaux sont ainsi directement accessibles et ne sont pas noyés dans le dossier contenant les résultats détaillés.

\section*{Crawler}
Les résultats globaux dans le dossier "logs" :
\begin{itemize}
	\item \textbf{stats\_flash-cookies.csv} contient la liste des sites utilisant des cookies Flash, triés par ordre décroissant.
	%\newline
\end{itemize}

\section*{Parser}
Les résultats globaux dans le dossier "logs" :
\begin{itemize}
	\item \textbf{stats\_detailed.csv} contient la liste des sites avec le nombre détaillé d'éléments enregistrés (trackers connus identifiés avec l'aide de Ghostery, réponses HTTP créant un cookie, fichiers JavaScript avec ou sans paramètres chargés depuis un autre domaine, Flash chargés depuis un autre domaine, pixels espions et requêtes d'URL avec des paramètres).
	\item \textbf{stats\_mimetypes\_ghostery.csv} contient la liste des types d'éléments (\textit{mimetype}) des trackers détectés par Ghostery, triés par ordre décroissant (si Ghostery a été utilisé par le \textit{parser}).
	\item \textbf{stats\_mimetypes\_soa.csv} contient la liste des types d'éléments (\textit{mimetype}) chargés d'un domaine différent, triés par ordre décroissant.
	\item \textbf{stats\_trackers.csv} contient la liste des trackers identifiés grâce à Ghostery, triés par ordre décroissant (si Ghostery a été utilisé par le \textit{parser}).
	\newline
\end{itemize}

Les résultats détaillés pour chaque site dans le dossier "results" :
\begin{itemize}
	\item \textbf{<URL du site>\_cookies.csv} contient la liste des cookies créés (avec leurs détails) par des réponses HTTP d'un domaine différent.
	\item \textbf{<URL du site>\_flash.csv} contient la liste des fichiers Flash provenant d'un autre domaine.
	\item \textbf{<URL du site>\_ghostery.csv} contient la liste des trackers détectés grâce à la base de données Ghostery (si Ghostery a été utilisé par le \textit{parser}).
	\item \textbf{<URL du site>\_js.csv} contient la liste des fichiers JavaScript provenant d'un domaine tiers.
	\item \textbf{<URL du site>\_js-query.csv} contient la liste des fichiers JavaScript provenant d'un domaine tiers et appelés avec des paramètres.
	\item \textbf{<URL du site>\_mimetypes.csv} contient la liste des types d'éléments (\textit{mimetype}) chargés d'un domaine différent, triés par ordre décroissant.
	\item \textbf{<URL du site>\_parameters.csv} contient la liste des requêtes vers un domaine différent dont l'URL contient des paramètres.
	\item \textbf{<URL du site>\_pixels.csv} contient la liste des pixels espions détectés depuis un autre domaine.
	\item \textbf{<URL du site>\_urls.csv} contient la liste de l'URL de toutes les ressources chargées d'un domaine différent.
	\newline
\end{itemize}

\chapter{Trackers détectés par Ghostery pour les principaux types MIME}
\label{ghostery_mimetypes}

\underline{\textbf{image/gif} (Images .gif) :}\\
 http://apx.moatads.com/pixel.gif?e=17\&i=AOL2\&cm=1\&bq=2\&f=[...]\\
 http://ums.adtech.de/mapuser/providerid=1037;userid=16c29fb9-8d[...]\\
 http://pixel.rubiconproject.com/tap.php?v=11581\&nid=2395\&put=[...]

\underline{\textbf{text/javascript} :}\\
 http://beacon-5.newrelic.com/1/a2134792db?a=163769\&ap=479\&be=1349\&fe=2[...]\\
 http://data103.adlooxtracking.com/ads/check/check.php?version=3\&client=ebuzz[...]\\
 http://rma-api.gravity.com/v1/beacons/initialize?u=undefined\&sg=95ec266b244d[...]

\underline{\textbf{application/x-javascript} :}\\
 http://at.atwola.com/addyn/3.0/5113.1/221794/0/-1/size=6x2;noperf=1;alias=93[...]\\
 http://tags.mathtag.com/notify/js?exch=adt\&id=5aW95q2jLzQvIC9ZVFl5Wmpk[...]\\
 http://assets.ebz.io/ebzFormats/buzzplayer/YoutubeBuzzPlayer.js?2.0.aab1d6fd2d[...]

\underline{\textbf{application/javascript} :}\\
 http://adadvisor.net/adscores/g.js?sid=9201023828\\
 http://j.adlooxtracking.com/ads/js/tfa\_ebuzz\_creaebz.js\\
 http://tacoda.at.atwola.com/rtx/r.js?cmd=ADG\&si=1808\&pu=http\%3A//www.h[...]

\underline{\textbf{text/html} (HTML) :}\\
 https://www.facebook.com/plugins/like.php?action=like\&app\_id=46744042133\&[...]\\
 http://ads.tw.adsonar.com/adserving/getAds.jsp?previousPlacementIds=\&placem[...]\\
 http://rma-api.gravity.com/v1/beacons/log?cbust=955-45\&site\_guid=95ec266b24[...]

\underline{\textbf{application/x-unknown-content-type} (Type inconnu) :}\\
 http://dtm.cb2.com/dmm/casale/match?fpc=2437\&pnid=19998\&trid=476302305[...]\\
 http://at.atwola.com/addyn/3.0/5113.1/221794/0/-1/noperf=1;alias=93314234;cf[...]\\
 http://pixel.quantserve.com/pixel;r=962460299;a=p-6fTutip1SMLM2;labels=[...]
