\appendix
\chapter{Options de l'outil implémenté}
Lors du lancement du \textit{crawler}, l'utilisateur peut spécifier plusieurs arguments :
\begin{itemize}
	\item Requis : le mode (\textit{crawler} ou \textit{parser}).
	\item Requis : le répertoire des fichiers.\\
		Pour le \textit{crawler}, il s'agit du répertoire où les fichiers seront enregistrés.\\
		Pour le \textit{parser}, il s'agit du répertoire contenant les fichiers à analyser.
	\item Optionnel : l'activation du mode \textit{debug} qui va donner davantage de détails en cas de problème.
	\item Optionnel : l'affichage de l'aide.
	\newline
	\item Requis pour le \textit{crawler} : le profil de Firefox à utiliser.
	\item Requis pour le \textit{crawler} : le fichier contenant la liste des sites web à visiter.
	\item Requis pour le \textit{crawler} : le début de l'intervalle des sites à visiter.
	\item Requis pour le \textit{crawler} : la fin de l'intervalle des sites à visiter.
	\item Requis pour le \textit{crawler} : le nombre maximal de tentatives par site web (\textit{timeout}).
	\item Requis pour le \textit{crawler} : le nombre de sites à visiter avant le redémarrage de Firefox.
	\newline
	\item Requis pour le \textit{parser} : le fichier contenant la liste des trackers de Ghostery.
	\item Optionnel pour le \textit{parser} : l'activation de l'impression de tous les trackers identifiés.
\end{itemize}
