\documentclass[compress]{beamer}
  \usepackage[utf8]{inputenc}
  \usepackage[frenchb]{babel}
  \usepackage[T1]{fontenc}
  %\usepackage[latin1]{inputenc}
  \usetheme{Warsaw}

  \title{Avancement mémoire}
  \author{Antoine Marchal}

  \begin{document}

  \begin{frame}
  \titlepage
  \end{frame}

  \section*{Plan}
  \begin{frame}
  \frametitle{Plan}
    \tableofcontents[]
  \end{frame} 

  \section{Présentation}
  \begin{frame}
  \frametitle{Présentation brève}
  La première étape de ce mémoire était d'identifier les moyens qui sont utilisés pour nous tracer sur Internet.
  
  Il se focalise sur le traçage au niveau du client (navigateur web) via notamment:
    \begin{itemize}
      \item Les cookies web
      \item Les scripts JavaScript
      \item Les images chargées depuis des sites tiers
      \item Flash
      \item ...
    \end{itemize}
  \end{frame}

  \begin{frame}
  \frametitle{Quels sites nous tracent ?}
  Afin de se rendre compte de l'étendue de ce traçage, l'élaboration d'un outil était nécessaire.
   \begin{itemize}
    \item Il est réalisé en Java
    \item Il se compose de deux modules principaux et indépendants:
      \begin{itemize}
        \item Crawler
        \item Parser
      \end{itemize}
    \end{itemize}
  \end{frame}
  
  \section{Crawler}
  \subsection{Implémentation}
  \begin{frame}
  \frametitle{Crawler}
    \begin{itemize}
      \item Il utilise Selenium afin de lancer une instance de Firefox
      \item Cette instance de Firefox est munie de deux extensions:
        \begin{itemize}
          \item Firebug : récupère tous les éléments chargés sur la page
          \item NetExport (extension de Firebug) : exporte au format HAR
        \end{itemize}
    \end{itemize}
  \end{frame}
  
  \subsection{Problèmes rencontrés}
  \begin{frame}
  \frametitle{Crawler : problèmes rencontrés}
    \begin{itemize}
      \item Selenium ne permettait pas de récupérer tous les éléments
      
      => J'ai alors mis en place un proxy
      \item Cependant, le proxy ne permettait pas d'exécuter du JavaScript et beaucoup de sites n'étaient alors pas traités correctement (entre 20 et 30\% des sites sur le TOP 100)
      
      => J'ai finalement décidé de supprimer le proxy et d'utiliser les extensions Firefox citées précédemment
    \end{itemize}
  \end{frame}
  
  \subsection{En pratique}
  \begin{frame}
  \frametitle{Crawler : en pratique}
  J'ai lancé le crawler sur le TOP 1000 et cela a pris environ 5h
    \begin{itemize}
    \item Sur une machine de la salle Intel
    \item Avec Xvfb (émulation de X)
    \item Une seule remarque : le stockage des fichiers occupe beaucoup d'espace disque
    \end{itemize}
  \end{frame}
  
  \section{Parser}
  \subsection{Implémentation}
  \begin{frame}
  \frametitle{Parser}
  Le parser permet de traiter les fichiers HAR et d'identifier les sites qui utilisent (potentiellement ou pas) des trackers.
  
  Il y a des aspects différents:
  \begin{itemize}
    \item Identifier les éléments chargés par le site et comparer avec une base de données de trackers (Ghostery*) => fiable
    \item Déterminer si les éléments chargés par le site sont stockés sur des serveurs tiers (utilisation des DNS) => moins fiable
    \item Identifier (provenance) les cookies créés sur le machine
    
  \end{itemize}
  
  *J'ai demandé à Ghostery la permission d'utiliser leur base de données de trackers.
  \end{frame}
  
  
  \subsection{Problèmes rencontrés}
  \begin{frame}
  \frametitle{Parser : problèmes rencontrés}
    \begin{itemize}
      \item Certains sites hébergent leur contenu sur un autre domaine
      
      => La solution est alors d'utiliser les DNS mais cela peut entraîner des faux positifs
    \end{itemize}
  \end{frame}
  
  \subsection{En pratique}
  \begin{frame}
  \frametitle{Parser : en pratique}
  J'ai lancé le parser sur le TOP 200 et cela a pris moins de 5 min
    \begin{itemize}
    \item Sur laptop personnel
    \item Au niveau de l'implémentation, seule la vérification par rapport à la base de données de trackers est terminée
    \end{itemize}
  \end{frame}
  
  \section{Suite...}
  \begin{frame}
  \frametitle{Suite...}
    \begin{itemize}
      \item Installer des extensions censées protéger la vie privée et relancer le programme sur cette instance particulière de Firefox 
      \item Regarder si le nombre de trackers varie par pays (TOP 100 de pays différents)
      \item Déterminer le pourcentage de sites se conformant au DNT
      \item Récupérer et identifier les cookies Flash
      \item ...
    \end{itemize}
  \end{frame}
  
  \section{Démo}
  \begin{frame}
  \frametitle{Démonstration}
  \end{frame}

  \end{document}
